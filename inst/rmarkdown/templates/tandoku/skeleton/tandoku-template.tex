% Setup Japanese Fonts
\usepackage{xeCJK}
\setCJKmainfont{Noto Sans CJK JP}
\setCJKmonofont{Noto Sans Mono CJK JP}
\setCJKsansfont{Noto Sans CJK JP}
\setromanfont[Mapping=tex-text]{Noto Serif}
\setsansfont[Scale=MatchLowercase,Mapping=tex-text]{Noto Sans}
\setmonofont[Scale=MatchLowercase]{Source Code Pro}
\XeTeXlinebreaklocale "ja"
\pagenumbering{gobble} % Prevent pagenmumbers

% Please add the following required packages to your document preamble:
\usepackage{tikz}
\usetikzlibrary{matrix, positioning}

\centering{\huge\textbf{学生単独出張理由書}}

\vspace{5em}

\begin{tikzpicture}[outer sep=0pt]
\node (A1) [minimum width = 3cm, minimum height = 3cm, draw, align=center]
{出張者 \\ 所属・学年・氏名};
\node (A2) [minimum width = 3cm, minimum height = 2cm, draw, below=0cm of A1.south west, anchor=north west, align=center]
{出張日程};
\node (A3) [minimum width = 3cm, minimum height = 2cm, draw, below=0cm of A2.south west, anchor=north west, align=center]
{用務先\\(所在地)};
\node (A4) [minimum width = 3cm, minimum height = 5cm, draw, below=0cm of A3.south west, anchor=north west, align=center]
{用務内容};
\node (A5) [minimum width = 3cm, minimum height = 2cm, draw, below=0cm of A4.south west, anchor=north west, align=center]
{依頼者が帯同\\できない理由};
\node (A6) [minimum width = 3cm, minimum height = 2cm, draw, below=0cm of A5.south west, anchor=north west, align=center]
{安全の確保};
\node (A7) [minimum width = 3cm, minimum height = 2cm, draw, below=0cm of A6.south west, anchor=north west, align=center]
{緊急連絡先};

\node (B1a)  [minimum width = 13cm, minimum height = 1.5cm, draw, right=0cm of A1.north east, anchor=north west] {
  \begin{minipage}{12cm}
  所属・専攻・学年等 \\
  `r SHOZOKU`
  \end{minipage}};

\node (B1b) [minimum width = 13cm, minimum height = 1.5cm, draw, below=0cm of B1a.south west, anchor=north west] {
  \begin{minipage}{12cm}
  `r NAME`
  \end{minipage}};

\node (B2) [minimum width = 13cm, minimum height = 2cm, draw, below=0cm of B1b, align=center] {
  \begin{minipage}{12cm}
  `r DATE`
  \end{minipage}};

\node (B3) [minimum width = 13cm, minimum height = 2cm, draw, below=0cm of B2, align=center] {
  \begin{minipage}{12cm}
  `r YOUMUSAKI`
  \end{minipage}};

\node (B4) [minimum width = 13cm, minimum height = 3cm, draw, below=0cm of B3, align=left] {
  \begin{minipage}{12cm}
  `r YOUMU`
  \end{minipage}};

\node (B5) [minimum width = 13cm, minimum height = 2cm, draw, below=0cm of B4, align=left] {
  \begin{minipage} {12cm}
  \leftskip 3em
  \parindent -3em
  注意)出張目的が大学(専攻、学科・コース、指導教員)の目標・計画に基づいたものである場合に限り、旅費として大学の経費(大学運営経費・寄附金等)を支出できます。
  \end{minipage}};

\node (B6) [minimum width = 13cm, minimum height = 2cm, draw, below=0cm of B5, align=left] {
  \begin{minipage} {12cm}
  `r REASON`
  \end{minipage}};

\node (B7) [minimum width = 13cm, minimum height = 2cm, draw, below=0cm of B6, align=left] {
  \begin{minipage} {12cm }
  `r SAFETY`
  \end{minipage}};

\node (B8) [minimum width = 13cm, minimum height = 2cm, draw, below=0cm of B7, align=left] {
  \begin{minipage}{12cm}
  `r CALLS`
  \end{minipage}};

\node (D1) [minimum width = 16cm, minimum height = 5cm, below=0cm of A7.south west, anchor = north west]{
  \begin{minipage}{15 cm}
  \vspace{1em}
  \parindent=1em
  上記のとおり、上記の学生の単独での出張が必要ですので届け出いたします。\\ \vspace{1em}

  出張者はこの用務を遂行するために必要な知識、能力を有する優秀な学生であり、教員の指導を受けずに用務を果たすことができます。

  またこのことにより本人の修学に支障が生じることはありません。\\
  \vspace{2em}
  \parindent=20em
  \indent 担当教員氏名・職名\\
  \indent Gregory N. Nishihara・准教授      印 \\
  \end{minipage}};

\end{tikzpicture}
